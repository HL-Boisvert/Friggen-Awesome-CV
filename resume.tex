%!TEX TS-program = xelatex
%!TEX encoding = UTF-8 Unicode
%%%%%%%%%%%%%%%%%%%%%%%%%%%%%%%%%%%%%%%%%
% Friggen-Awesome CV: XeLaTeX template for a CV
%
% https://github.com/jagmoreira/Friggen-Awesome-CV
% Author: Joao Moreira (https://github.com/jagmoreira)
%
% %% Body based on:
% CV Friggeri X version 1.0 (01/01/2016) by Nadorrano
% https://github.com/Nadorrano/cv-friggeri-x
%
% %% Header based on:
% Awesome CV by Claud D. Park
% https://github.com/posquit0/Awesome-CV
%
% License:
% CC BY-NC-SA 4.0 (http://creativecommons.org/licenses/by-nc-sa/4.0/)
%
% Important notes:
% This template needs to be compiled with XeLaTeX and the bibliography,
% if used, needs to be compiled with biber rather than bibtex.
%
%%%%%%%%%%%%%%%%%%%%%%%%%%%%%%%%%%%%%%%%%

\documentclass[12pt]{friggen-awesome-cv}
% Add `a4paper` to set a4 paper size
% Add `nocolors` to remove colors from the document
% Add `sidebar` to enable a sidebar section on the left

% Configure page margins to properly fit your content
\geometry{left=1cm,top=3cm,bottom=0.5cm,right=1cm,headsep=10pt,nofoot}

%-------------------------------------------------------------------------------
%   PERSONAL INFORMATION
%-------------------------------------------------------------------------------
 

\begin{document}

\header{Jo\~{a}o}{Moreira}% Your name

%-------------------------------------------------------------------------------
%	SIDEBAR SECTION
%-------------------------------------------------------------------------------
% If `sidebar` is added in the documentclass definition above, uncomment these
% lines to allow a sidebar on the left. See template section
%\begin{aside}
%%-------------------------------------------------------------------------
%   TECHNICAL SKILLS SECTION
%-------------------------------------------------------------------------
\section{Skills}
\subsection{ML libraries}
TensorFlow
Keras
scikit-learn
pandas
fastai
~
\subsection{Development}
Python~ – ~3 yrs
Java~ – ~3 yrs
R~ – ~1 yr
~
\subsection{Databases}
MySQL
MongoDB
Neo4j
Airflow \& Kafka
Azure
~
\subsection{Others}
Git
\LaTeX
MATLAB/Simulink
~
\subsection{Languages}
French ~ – ~ Native
English ~ – ~ Fluent
German ~ – ~ Novice%
%\end{aside}
% %
% % Leave the percent signs to avoid creating a paragraph between minipages
%
%-------------------------------------------------------------------------------
%   MAIN SECTION
%-------------------------------------------------------------------------------
\begin{main}
%-------------------------------------------------------------------------
%   EDUCATION SECTION
%-------------------------------------------------------------------------

\section{education}

\begin{entrylist}

%------------------------------------------------

\shortentry
{Ph.D.}
{Chemical Engineering, Northwestern University}
{2012--2017}
{Thesis: \textit{The Effect of Gender Diversity in Creative Teams}}

%------------------------------------------------

\shortentry
{M.Sc.}
{Physics, University of Lisbon, Portugal}
{2009--2011}
{Thesis: \textit{Evolutionary Dynamics of Cooperation in Multiplayer Games}}

%------------------------------------------------

\shortentry
{B.Sc.}
{Physics, University of Lisbon, Portugal}
{2006--2009}
{}

%------------------------------------------------

\end{entrylist}

%-------------------------------------------------------------------------
%   SKILLS SECTION
%-------------------------------------------------------------------------

\section{skills}

\begin{entrylist}

%------------------------------------------------

\minientry
{Data analysis}
{Machine learning, time-series analysis, maximum likelihood, bootstrapping, Monte Carlo methods, map-reduce}

%------------------------------------------------

\minientry
{Programming}
{Python (including numpy, scipy, pandas, scikit-learn, requests), R, BASH, C++, git}

%------------------------------------------------

\minientry
{Databases}
{MySQL, MongoDB}

%------------------------------------------------

\minientry
{Web design}
{HTML, CSS, Javascript, D3.js}

%------------------------------------------------

\minientry
{Languages}{Portuguese (native), English (fluent)}

\end{entrylist}

%-------------------------------------------------------------------------
%   WORK EXPERIENCE SECTION
%-------------------------------------------------------------------------

\section{experience}

\begin{entrylist}

%------------------------------------------------

\entry
{2011--Present}
{Northwestern University}
{Evanston, IL}
{\textit{\underline{Data Scientist}}\\
\begin{itemize}
\item Used maximum likelihood estimation to parameterize the distribution of the impact of scientific publications (\textasciitilde1TB dataset)
\item Created a Django web application to showcase my research: \href{http://forecite.us}{http://forecite.us}
\item Performed time-series analysis on IMDb metadata of +15k movies
\end{itemize}
\textit{\underline{Instructor}}\\
\begin{itemize}
\item Developed course materials and was an instructor for \href{https://amarallab.github.io/Introduction-to-Python-Programming-and-Data-Science/}{`Introduction to Python Programming and Data Science Bootcamp'}
\item TA for Process Control, Quantum Mechanics, Mass Transfer
\item Mentored 2 younger graduate students and 2 high-school interns
\end{itemize}}

%------------------------------------------------

\entry
{2016 | 6 mos}
{\href{http://gadflyzone.com/}{GadflyZone}}
{Vernon Hills, IL}
{\textit{\underline{Data Science Intern}}\\
\begin{itemize}
\item Constructed MySQL database of patents on Amazon Database Relational Services built from publicly available patent information
\item Used natural language processing and citation analysis to design a multidimensional scoring engine to predict a company's innovativeness
\end{itemize}}

%------------------------------------------------

\entry
{2014 | 4 mos}
{\href{https://narrativescience.com/}{Narrative Science}}
{Chicago, IL}
{\textit{\underline{Software Engineer}}\\
\begin{itemize}
\item Developed Python tools to automatically gather and perform statistical analysis of stories written by Quill, Narrative Science's patented artificial intelligence platform
\item Used Quill to collect the most relevant metrics from the stories' analysis and to generate an email report that produced insight about the inner workings of the platform
\end{itemize}}

%------------------------------------------------

\end{entrylist}

\end{main}

\end{document}
